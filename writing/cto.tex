%% Default Latex document template
%%
%%  blake@rcs.ee.washington.edu

\documentclass[letterpaper]{article}

% Uncomment for bibliog.
%\bibliographystyle{unsrt}

\usepackage{graphicx}
\usepackage{lineno}
\usepackage{amsmath}
%\usepackage{fancyhdr}

%%%%%%%%%%%%%%%%%%%%%%%%%%%%%%%%%%%%%%%%5
%
%  Set Up Margins
\input{templates/pagedim.tex}

%
%        Font selection
%
%\renewcommand{\rmdefault}{ptm}             % Times
%\renewcommand{\rmdefault}{phv}             % Helvetica
%\renewcommand{\rmdefault}{pcr}             % Courier
%\renewcommand{\rmdefault}{pbk}             % Bookman
%\renewcommand{\rmdefault}{pag}             % Avant Garde
%\renewcommand{\rmdefault}{ppl}             % Palatino
%\renewcommand{\rmdefault}{pch}             % Charter


%%%%%%%%%%%%%%%%%%%%%%%%%%%%%%%%%%%%%%%%%%%%%%%%%
%
%         Page format Mods HERE
%
%Mod's to page size for this document
\addtolength\textwidth{0cm}
\addtolength\oddsidemargin{0cm}
\addtolength\headsep{0cm}
\addtolength\textheight{0cm}
%\linespread{0.894}   % 0.894 = 6 lines per inch, 1 = "single",  1.6 = "double"

% header options for fancyhdr

%\pagestyle{fancy}
%\lhead{LEFT HEADER}
%\chead{CENTER HEADER}
%\rhead{RIGHT HEADER}
%\lfoot{Hannaford, U. of Washington}
%\rfoot{\today}
%\cfoot{\thepage}



% Make table rows deeper
%\renewcommand\arraystretch{2.0}% Vertical Row size, 1.0 is for standard spacing)

\begin{document}
\section*{Path Optimization: Brute Force Search}

\section{Basics}

\subsection{Notation}
The goal is to search a grid of points in the  space consisting of points  $P_i = \{X_i, V_i\} = \{x,y,z,\dot{x},\dot{y},\dot{z}\}$ within bounds:
\[
-1 < x < 1, \;
-1 < y < 1, \;
-1 < z < 1, \;
-1 < \dot{x} < 1, \;
-1 < \dot{y} < 1, \;
-1 < \dot{z} < 1, \;
\]


We wish to visit all the points with as low a cost as possible.
We will set up a grid with $N$ points per axis, for a total of $6^N$ points.

A {\it trajectory}, $T_{ij}$ between two points in this space, $T(P_i,P_j)$, is a route through
the space from $P_i$ to $P_{i+1}$ with the properties

\beq \label{firstconstraint}
\Delta X (T_{ij}) = \frac{X_{i+1}-X_i}{\Delta t}
\eeq
%
% \beq
% | \frac {\Delta V} {\Delta t} |_\infty  < a_{max}
% \eeq
% \beq
% | \frac {\Delta V} {\Delta t} |_\infty < v_{max}
% \eeq
% \beq \label{complexconstraint}
% |\Delta V +  \frac {\Delta X}{\Delta t} |_\infty < v_{max}
% \eeq

% Figure 1
\begin{figure}\centering
%   \includegraphics[width=3.0in]{basicTraj.png}
  \includegraphics[width=\textwidth]{computedTraj.png}
  \caption{}\label{basicTraj}
\end{figure}

% Constraint eqn (\ref{complexconstraint}) can be seen from Figure \ref{basicTraj}.
%
% Two components of ${V}_i$
% are shown, one arising from $\Delta X$ and another arising from $\Delta V$.


\subsection{Trajectories between points}

A {\bf Trajectory}, $T_{ij}$ connects point $P_i$ to point $P_j$
in phase space with a time function $x(t), 0<t<\Delta t $.  To meet the constraints
\beq \label{trajConstraints}
x(0) = X_i, \; x(\Delta t) = X_j, \; \dot{x}(0) = V_i, \;
\dot{x}(\Delta t) = V_j
\eeq
we can use a 3rd order polynomial having four unknown constants:
\beq
\begin{aligned} \label{polyTraj}
x(t) & = a_0 + a_1t + a_2t^2 + a_3t^3 \\
v(t) & = \hspace{7mm}  a_1+2a_2t + 3a_3t^2 \\
a(t) & = \hspace{16mm}  2a_2 + 6a_3 t \\
\end{aligned}
\eeq

A typical trajectory of this type, computed for
\beq
x(0) =  -2, \; v(0) = -1, \qquad  x(\Delta t) = 1.5, \; v(\Delta t) = 1.5
\eeq
is given in Figure \ref{basicTraj}.

The constants are solved as follows:
\beq
a_0 = x(0),  a_1 = v(0)
\eeq
defining some intermediate terms:

\beq
\Delta x = x(\Delta t)-x(0) \qquad \Delta v = v(DT)-v(0)
\eeq

\beq
b0 = \Delta t \qquad b1 = {\Delta t}^2 \qquad b2 = {\Delta t}^3
\eeq

\beq
b3 = 2\Delta t \qquad b4 = 3{\Delta t}^2
\eeq
then
\beq
a_3 = \frac {b1 \Delta v - b3(\Delta x-v(0)b0)}  {b1b4-b2b3}
\eeq
\beq
a_2 = \frac  {\Delta x - v(0)b0 - a_3b2}   {b1}
\eeq
Our goal is to find a minimum cost trajectory satisfying eqn
(\ref{trajConstraints}) and with the form of eqn (\ref{polyTraj}).
Then we can define the cost of each trajectory between two phase-space points at least two ways:

\subsubsection{Energy Cost}    We assume that energy of a trajectory is
\beq
C_{e}(T_{ij}) = \int_0^{\Delta t} a(t)^2 dt
\eeq

\subsubsection{Duration Cost}   The time cost, $C_t$ is
\beq
C_t(T_{ij}) = \Delta t
\eeq

\subsubsection{Acceleration Constraint}

To assure that our trajectories, $x(t),v(t),a(t)$ are feasible
for a real robot manipulator, we will constrain
\beq
a(t) < a_{max} \qquad 0<t<\Delta t
\eeq

furthermore we wish to complete the trajectory as fast as
feasible, so we will set this constraint to equality:
\beq
\max(|(a(t)|) = a_{max}
\eeq

From eqn \ref{polyTraj} we know that acceleration is linear with
time for all solutions, thus we have:
\beq \label{acc_max}
\max(|a(t)|) = \max(a(0), a(\Delta t) )
\eeq

We iteratively minimize $\Delta t$ for each trajectory until
eqn(\ref{acc_max}) is satisfied.

\subsection{Path Cost}
A {\it path}, $\mathbf{P}$, is a sequence of
trajectories (indexed by $k$), $T_{ijk}$,
connecting $P_i$ to $P_j$
such that the trajectories are connected, e.g.
\beq
P_j(T_{ijk}) = P_i(T_{ijk+1})
\eeq

the points $P_i$ covering the entire grid.
Let $C_i=C_x(T_{ijk}$ be the cost of the $kth$ trajectory in
the path, $\mathbf{P}$.
The time   cost of visiting every point in the path is
\beq
C_T = \Sigma_i C_i  \qquad 0 \leq i < 6^N
\eeq
For example, the total duration cost of path ${P}_1$ would be
\beq
C_{Tp} = \Sigma_k C_t(T_{ijk})
\eeq
where $T_{ijk}$ is the $k^{th}$ trajectory of ${P}_1$.

\section{Problem Statement}

We can now state our problem as, given a set of points defining a uniform grid in a position and velocity
space, with $N$ points between bounds $\{-1,1\}$ in each dimension, find the path $\mathbf{P}_{opt}$
that visits all the points with the lowest overall cost.   This problem is similar to the famous
Traveling Salesman problem, but the map is simplified by a grid patterns and our coordinates are coupled by
the fact that
\beq
X(t) = \int_0^t V(t) dt
\eeq
Because of the constraints (eqns \ref{firstconstraint} to \ref{complexconstraint}), this reduces to
choosing the sequence of points visited.


\section{Brute force searching}

\subsection{1D example}
We start with a simple case of a two dimensional space consisting of a scalar position and velocity.  We grid the
space with $N=4$ points in position and velocity for a total of 16 points.

Some possible trajectories are shown in Figure \ref{handsolutions1D}.

% Figure 2
\begin{figure}\centering
  \includegraphics[width=3.0in]{handTraj01.png}
  \includegraphics[width=3.0in]{handTraj02.png}
  \caption{}\label{handsolutions1D}
\end{figure}

The two trajectories both start at $X=-1.0$ and $\dot{X}=0.33$
(larger green dot).  The first path minimizes energy cost
and the second minimizes total time.  Both are subject to the
acceleration constraint.

\subsection{simulation structure}
We define the following classes:
\begin{itemize}
  \item {\tt grid}   The grid and associated methods.
  \item {\tt point}  A single point and associated methods.
  \item {\tt trajectory} A trajectory between two points and methods for time-evolution, cost computation.
\end{itemize}

\subsection{overall strategy}

\begin{enumerate}
  \item Create the grid and associated points
  \item Compute for each point, compute the cost of a trajectory between it and all the other points.  Populate
  a cost matrix with $N^6$ rows and $N^6$ columns where
  \beq
    Cm_{ij} = C_x(T_{ij})  \qquad    x \in \{e,t\}
  \eeq
  \item Manually select starting point ($P_0$)
  \item Select the lowest cost column from the starting point row of $Cm$.
  \item Block that point (marking a list)
  \item Repeat until all points are blocked.
\end{enumerate}


\section{Research Questions}


\subsection{Global optimality}\label{prob:global}

This version of the traveling salesman problem is clearly asymmetric as it is
easy to find pairs of phase space points for which
\beq
C_x (P_1,P_2) \neq C_x (P2,P1)
\eeq
and it is known \cite{}  that greedy algorithms can perform very badly for asymmetric or
non-Euclidean graphs.

For small problem domains we can find the global optimumum Hamiltonian tour
through a brute force search and compare it with our heuristic.

\paragraph{Problem \ref{prob:global}}
For the 2D rectangular grid in phase space (1D position, 1D velocity),
find the optimal path via brute force search for all possible starting points
and compare its cost to
one or more greedy searches.

\subsection{Starting point dependence}\label{prob:startingpt}
Huristic greedy searches produce a result which can depend on starting point.
In our application, a robot must be initialized and then incur a small cost
(compared to the overall trajectory cost) to more to the starting point
selected for an optimal path.

\paragraph{Problem \ref{prob:startingpt}}
For both the 2D and 6D grids, initialize the heuristic at several starting
points and compute
mean and variance of total path cost.



\end{document}

